\documentclass{beamer}

\usepackage{graphicx, amsmath, amsthm, amssymb, tikz, listings}

\lstset{language=C, frame=single}

\definecolor{UDBlue}{RGB}{0,83,159}
\definecolor{UDGold}{RGB}{255,210,0}

\usetheme{Madrid}
\usecolortheme[RGB={0,83,159}]{structure}
\setbeamertemplate{blocks}[rounded][shadow=true]
\useoutertheme{split}
\beamertemplatenavigationsymbolsempty

\author{Ryan McKenna, Matthew Paul, James Kerrigan \\
Niko Gerassimakis, Neil Duffy }
\title{Team 6: LU Factorization}
\subtitle{Optimizations targeting towards multicore processors}
\date{\today}

\logo{
	\includegraphics[height=1cm, keepaspectratio]{delaware_logo.png}
}

\begin{document}

\frame{\maketitle}

\begin{frame}

\frametitle{Linear Algebra}

\begin{itemize}
\item The quintessential problem in linear algebra is solving a linear system of equations

$$
\begin{bmatrix}
	a_{11} &  a_{12} &  a_{13}  \\
	a_{21}  &  a_{22} &  a_{23}  \\
	a_{31}  &  a_{32}  &  a_{33}
\end{bmatrix}
\begin{bmatrix}
x_1\\x_2\\x_3
\end{bmatrix}
=
\begin{bmatrix}
b_1\\b_2\\b_3
\end{bmatrix}
$$

\item We want to find values of $ x_1, x_2, $ and $ x_3 $ such that 

\begin{enumerate}
\item $ a_{11} x_1 + a_{12} x_2 + a_{13} x_3 = b_1 $
\item $ a_{21} x_1 + a_{22} x_2 + a_{23} x_3 = b_2 $
\item $ a_{31} x_1 + a_{32} x_2 + a_{33} x_3 = b_3 $
\end{enumerate}

\end{itemize}

\end{frame}


\begin{frame}
\frametitle{Impact}

\begin{itemize}
\item Linear algebra comes up in a lot of professions:
\begin{itemize}
\setlength\itemsep{0.25em}
\item Physics
\item Partial differential equations
\item Graph theory
\item Statistics / Curve Fitting
\item Sports Ranking
\end{itemize}

\end{itemize}
	
\end{frame}


\begin{frame}
\frametitle{Solving Linear Systems}

\begin{itemize}
\item If $A$ is an $ n $ x $ n $ matrix, solving a system of the form $ A x = b $ takes $ O(n^3) $ time.

\item If $ A$ is a triangular matrix, then solving the system takes $ O(n^2) $ time
\end{itemize}
\end{frame}

\begin{frame}
\frametitle{LU Factorization Background}

\begin{itemize}
\item LU Factorization works by decomposing a square matrix $A$ into a lower triangular matrix, $L$, and an upper triangular matrix, $U$:

$$ A = L U $$

$$ 
\begin{bmatrix}
	a_{11} &  a_{12} &  a_{13}  \\
	a_{21}  &  a_{22} &  a_{23}  \\
	a_{31}  &  a_{32}  &  a_{33}
\end{bmatrix}
=
\begin{bmatrix}
	l_{11} &  0 &  0  \\
	l_{21}  &  l_{22} &  0  \\
	l_{31}  &  l_{32}  &  l_{33}
\end{bmatrix}
\begin{bmatrix}
	u_{11} &  u_{12} &  u_{13}  \\
	0  &  u_{22} &  u_{23}  \\
	0  &  0  &  u_{33}
\end{bmatrix}
$$

\item With $ L $ and $ U $, we can solve $ A x = L U x = b $ in $ O(n^2) $.  


\end{itemize}
\end{frame}

\begin{frame}
\frametitle{Algorithm Description / Access Pattern}

LU factorization is an $ O(n^3) $ algorithm:

\begin{center}
\includegraphics[scale=0.2]{figures/lu1a}\hspace{1em}
\includegraphics[scale=0.2]{figures/lu1b}
\hspace{1em}
\includegraphics[scale=0.2]{figures/lu1c}
\hspace{1em}
\includegraphics[scale=0.2]{figures/lu2} 


\includegraphics[scale=0.2]{figures/lu3}
\hspace{1em}
\includegraphics[scale=0.2]{figures/lu4}
\hspace{1em}
\includegraphics[scale=0.2]{figures/lu2b}
\hspace{1em}
\includegraphics[scale=0.2]{figures/lu5}
$$\vdots$$
\includegraphics[scale=0.2]{figures/lu6}
$$\vdots$$
\includegraphics[scale=0.2]{figures/lu7}
\end{center}

\end{frame}


\begin{frame}[fragile]
\frametitle{Implementation}

\begin{lstlisting}
void lu(double **A, double **L, double **U, int n) {
    zero (L, n);
    copy (U, A, n);
    init (L, n);
    for(int j=0; j < n; j++) {
        for(int i=j+1; i < n; i++) {
            double m = U[i][j] / U[j][j];
            L[i][j] = m;
            for(int k=j; k < n; k++)
                U[i][k] -= m*U[j][k];
        }
    }
}
\end{lstlisting}

\end{frame}

\begin{frame}
\frametitle{Approach}
\begin{enumerate}
\item Generate random matrices up to 6400 x 6400.
\item Run and time 4 trials of the factorization algorithm for each matrix size.
\item Repeat for every optimization configuration.
\end{enumerate}
\end{frame}

\begin{frame}
\frametitle{Optimizations}

\begin{itemize}
\item -O1, -O2, -O3
\item loop unrolling
\item vectorization
\item native (architecture specific) optimizations
\item openMP
\end{itemize}

\end{frame}

\end{document}